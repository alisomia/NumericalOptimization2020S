\documentclass{article}
\usepackage[ruled, linesnumbered]{algorithm2e}
\def\showtopic{Numerical Optimization}
\def\showtitle{Lab 1: Newton and Quasi-Newton Methods}
\def\showabs{Lab 1}
\def\showauthor{Ting Lin, 1700010644}
\usepackage{amsmath, amsfonts, amsthm}

\usepackage{graphicx, epstopdf}
\usepackage{color}
\usepackage{geometry, graphicx}
\usepackage{algorithm, algorithmic}
\usepackage{bm}
\usepackage{multirow}
\usepackage{ulem}
\geometry{left = 5em, right = 5em}
\usepackage{listings}
\usepackage{xcolor}
%% notation macro
\newcommand{\F}{\mathcal F}
\newcommand{\T}{\mathcal T}
\newcommand{\I}{\mathcal I}
\newcommand{\U}{\mathcal U}
\newcommand{\R}{\mathbb R}
\renewcommand{\P}{\mathcal P}
\newcommand{\uP}{ \mathcal \uline P}
\newcommand{\B}{\mathcal B}
%\newcommand{\R}{\mathbb R^2}
\newcommand{\Z}{\mathbb Z}
\newcommand{\C}{\mathbb C}
\newcommand{\laplacian}{\triangle}
\newcommand{\grad}{\nabla}
\renewcommand{\div}{\textrm{div~}}

\newcommand{\diff}[2]{\frac{\partial #1}{\partial #2}}
\newcommand{\difff}[3]{\frac{\parial #1^2}{\partial #2 \partial #3}}
\newcommand{\diFF}[2]{\frac{\partial #1^2}{\partial^2 #2}}
\newcommand{\diam}{\text{ diam }}
%% non-noation macro
\newcommand{\IN}{\text{  in  }}
\newcommand{\ON}{\text{  on  }}
\newcommand{\st}{\text{s.t.  }}
\newcommand{\tbc}{{\color{red}[TBC]}}
\renewcommand{\Return}{\textbf{return~}}
\newcommand{\Break}{\textbf{break~}}
\newcommand{\Continue}{\textbf{continue~}}
\renewcommand{\And}{\textbf{~and~}}
\newcommand{\Or}{\textbf{~or~}}
%% enviorment
\newtheorem{proposition}{Proposition}
\newtheorem{definition}{Definition}
\newtheorem{corollary}{Corollary}
\newtheorem{remark}{Remark}

\DeclareMathOperator{\argmin}{arg~min}


\title{\textbf{\showtitle}}
\author{\showauthor}
\usepackage{indentfirst}
\usepackage{fancyhdr}  
\pagestyle{fancy}
\lhead{\textbf {\showtopic} }
\chead{\showchead} 
\rhead{\textbf {\showabs} }
\lfoot{} 
\cfoot{\thepage}
\rfoot{} 
\renewcommand{\headrulewidth}{0.4pt} 
\newcommand{\return}{\textbf{return~}}
\DeclareMathOperator{\argmin}{arg~min}
\begin{document}
	\maketitle
	\thispagestyle{fancy}
	\tableofcontents
	
	\section*{}
In this section we introduce several Newton and Quasi-Newton methods, and some of their variants, with necessary numerical experiments. Before the concrete algorithms, we first need to introduce the basic framework of optimization, and some methods in line search. Then we introduce Newton methods, typically damped Newton (i.e. Newton with line search) with two modifications(so-called \textbf{mixed} and \textbf{LM}). In the third part we introduce Quasi-Newton method, which reduce the requirement of excessive computational complexity on Hessian. We end this report by showing numerical experiments on some examples with different scales, testing the efficiency and difference of each method. For notation convention, we use $f,g,h$ to denote the target, gradient and hessian if no ambiguity.

\section{Basic Framework and Line Search}
\subsection{Basic Framework of Optimization}
The basic framework of optimization discussed in this report are as follows:
\begin{algorithm}[H]
\label{alg:framework}
	\caption{Framework of optimization method}
	\KwIn{$f$, $x_0$}
	\KwOut{$x_{out}$}
	Given initial Point $x_0$, $k=0$
	
	\While{stop critertion does not meet}{
	{Compute the descent direction $d_k$}
	
	\eIf{Inexact Line Search return $\alpha_k$ sucessfully}{$x_{k+1} = x_k + \alpha_kd_k$}{Exact Line Search to obtain $x_{k+1}$}
	
	$k = k+1$}
	
\end{algorithm}
Usually the stop criterion might be chosen as $|x_{k} - x_{k-1}| \le tol$, or $|g_{k}| \le tol$ or $|f(x_k) - f(x_{k-1})| \le tol$. In this report we choose the first criterion in implementation.
We first introduce the line search method. By setting $\phi(a) = f(x_k + ad_k)$
\subsection{Exact Line Search Method: .618 method}
We first introduce the .618 method for exact line search. An auxiliary algorithm is \textbf{Find-Initial-Interval}, aiming to find a initial interval $l,r$ such that there exists an $m \in (l,r)$, and $f(m) < f(l),f(r)$.
\begin{algorithm}[H]
	\DontPrintSemicolon
	\label{alg:find-initial-interval}
	\caption{Find-Initial-Interval}
	\KwIn{$\phi$ \tcp*{$\phi$ need to be coerocive.}}
	\KwOut{Initial Search Interval: $(l,r)$}
	$x = 1$\\
	\eIf{$\phi(x) < \phi(0)$}{
		\While{$\phi(x)<\phi(0)$}{$x = 2x$} \return $(0,x)$
		}{\eIf{$\phi(-x)<\phi(0)$} {
			\While{$\phi(-x)<\phi(0)$}{$x = 2x$} \return $(-x,0)$}
		{\return $(-1,1)$}}
\end{algorithm}
Clearly, Algorithm~\ref{alg:find-initial-interval} will stop if $\phi$ is coerocive, i.e. $$\lim_{|x|\to\infty}\phi(x) = \infty$$. Here and throughout this report, we will assume our function is coerocive, so are our numerical experiments.

Based on the algorithm showed above, we introduce the .618 method.
\begin{algorithm}[H]

	\caption{.618 method}
	\KwIn{$\phi$ \tcp*{$\phi$ need to be coerocive.}}
	\KwOut{Stepsize: $\alpha$}
	$(l,r)$ = Find-Initial-Interval($\phi$)\\
	\While{$r - l < tol$}
	{$m_1  .618*l + .382*r, \qquad m_2 = .618*r + .382*l$\\
		\eIf{$m_1 > m_2$}{$l=m_1$}{$r=m_2$}
	}
\return $\frac{l+r}{2}$
\end{algorithm}
\subsection{Inexact Line Search Method}
However, exact line search method is too expansive for practical applications, hence we need a more efficient method. The \textbf{inexact line search method} can be thought as a great alternative.
We give three type of inexact line search method: backtracking and zoom(interpolation).
\begin{algorithm}[H]
	\caption{Backtracking}
	\KwIn{$\phi$}
	\KwOut{$\alpha$}
	Initialize $\alpha$
	
	$iter = 0$
	
	\While{ Rule($\alpha$) is False and iter $<$ MAXITER}
	{$\alpha = 0.9*\alpha$
	
iter = iter + 1}
\end{algorithm}

The $Rule(\alpha)$ can be chosen as one of following (Here $g_k = g(x_k)$)):
\begin{itemize}
	\item Armijo: $\phi(\alpha) - \phi(0) \le \rho\alpha g_k^Td_k$
	\item Goldstein:  $(1-\rho)\alpha g_k^Td_k \le \phi(\alpha) - \phi(0) \le \rho\alpha g_k^Td_k$
	\item Strong Wolfe: Armijo + $|g(\alpha)^Td_k| \le -\sigma g_k^Td_k$
	\item Wolfe: Armijo +  $g(\alpha)^Td_k \ge \sigma g_k^Td_k$
\end{itemize}

Next we introduce the zoom method, which is based on interpolation. 

\begin{algorithm}[H]
	\caption{Backtracking}
	\KwIn{$\phi$}
	\KwOut{$\alpha$}
	Initialize $\alpha$
	
	$iter = 0$
	
	\While{ Rule($\alpha$) is False and iter $<$ MAXITER}
	{Generate $\alpha^+$ from $\alpha$ by interpolation
		
		iter = iter + 1
	
$\alpha = \alpha^+$}
\end{algorithm}
Here we only consider the quadratic interpolation, that is, to find a quadratic polynomial $p(x)$, such that $$p(0) = \phi(0), p(\alpha) = \phi(\alpha), p'(0) = \phi'(\alpha).$$ Then $$\alpha^+ := \argmin_{\alpha} p(\alpha).$$ Direct calculation shows 
$$\alpha^+ = \frac{2\alpha^2\phi'(0)}{\phi(\alpha) - \phi(0) - \phi'(0)\alpha}.$$


\section{Newton Method}
\subsection{Naive Newton Method}
Naive Newton method is just a naive Newton method, utilizing the Hessian information directly. 
The following algorithm shows how to get the newton direction.
\begin{algorithm}
	\DontPrintSemicolon
	\label{alg:newton}
	\caption{Naive Newton}
	\KwIn{$f$, $x_0$}
	\KwOut{$x_{out}$}
	k = 0
	
	\While{ not converge}{
	Calculate $H = h(x_k)$, $g = g(x_k)$, here $h$ and $g$ is hessian and gradient.
	
	$x_{k+1} = x_{k}-H^{-1}g$ \tcc*{Suppose $H$ is invertible here}
	
	$ k = k+1$
}

\return $x_{k-1}$
\end{algorithm}
It can be proved that if the initial value is close to the local minimum $x*$, then Newton method yields second-order convergence.

\subsection{Damped Newton Method}
However, the above Newton Method might be too bold, hence we apply a damped technique by using line search after getting each Newton method. Here we return to Algorithm~\ref{alg:framework} with direction being normalized Newton direction
$$d_k = \frac{d_k^N}{|d_k^N|},$$
where $d_k^N$ means Newton direction in Algorithm~\ref{alg:newton}.
\subsection{Modification}
We introduce two modification to avoid that the Newton direction is not a descent direction, otherwise the line search method would fail. Two simple strategies is adopted in the report and the implementation, as listed below. 
\begin{algorithm}
	\caption{Newton-direction-mixed}
	\DontPrintSemicolon
	\KwIn{$f$, $x$}
	\KwOut{$d$}
		Calculate $H = h(x_k)$, $g = g(x_k)$  \tcc*{If $g=0$, we are done!}
		
		\If{$H$ is not invertible}{\return $-g/|g|$}
		Calculate $d = -H^{-1}g$
		
		\If{$d^Tg>0.3|d||g|$}{\return $-d/|d|$}
		\If{$d^Tg<-0.3|d||g|$}{\return $d/|d|$}
		\return $-g/|g|$.
		
\end{algorithm}

\begin{algorithm}
	\caption{Newton-direction-LM}
	\DontPrintSemicolon
\KwIn{$f$, $x$}
\KwOut{$d$}
Calculate $H = h(x_k)$, $g = g(x_k)$  \tcc*{If $g=0$, we are done!}

$\nu = 1$

$d = -H^{-1}g$

\While{$H$ is not invertible or $d^Tg > -0.3|d||g|$}{$d = -(H+\nu I)^{-1}g$

$\nu = 2\nu$}
\end{algorithm}
These two algorithm remedy the naive Newton by let the actual direction being between Newton and minus-gradient.  The mixed method chooses either one of them, and LM method changes more gently. However, this will force the Newton method have only first-order convergence rate.


\section{Quasi-Newton Methods}

In this section we introduce the quasi-Newton method, which only need compute Hessian once. We compute $H_0 = H(x_0)$ in advance. Here we omit the detailed deduction but only show the result in our algorithm. The first algorithm is use the objective value and gradient value to determine $H_{k+1}$, known as Broyden's family.

\begin{algorithm}
	\DontPrintSemicolon
	\caption{Broyden's family}
	\KwIn{$x_k, x_{k+1},H_{k}$}
	\KwOut{{$H_{k+1}$}}
	$g_k = g(x_k),~~ g_{k+1} = g(x_{k+1})$
	
	$s = x_{k+1} - x_k, ~~ y = g_{k+1} - g_k$
	
	
	\KwOut{$H_{k+1}$}
	
	$$H^{DFP} = H_k + \frac{s_ks_k^T}{s_k^Ty_k} - \frac{Hky_ky_k^TH_k}{y_k^TH_ky_k}$$
	
	$$H^{BFGS} = H_k + (1+\frac{y_k^TH_ky_k}{y_k^Ts_k})\frac{s_ks_k^T}{y_k^Ts_k} - (\frac{s_ky_k^TH_k + H_ky_ks_k^T}{y_k^Ts_k})$$
	
	\return $H_{k+1} = H^{DFP} + \varphi(H^{BFGS} - H^{DFP})$
\end{algorithm}

\begin{algorithm}
	\DontPrintSemicolon
	\caption{Quasi-Newton direction}
	\KwIn{$x_k, x_{k+1},H_{k}$}
	\KwOut{$d_{k+1}$}
	
	Obtain $H_{k+1}$ by Broyden's family
	
	Replace the true Hessian by $H_{k+1}$ in Newton-direction-mixed or Newton-direction-LM, obtaining $d_{k+1}$.
	
	\return $d_{k+1}$.

\end{algorithm}

Several special cases should be mentioned. If $\varphi = 0$, we obtain DFP formula, if $\varphi = 1$ we obtain BFGS formula. They are both Rank 2 modification. Interestingly, if we choose $\varphi = xx$ we obtain the SR1 formula, it is an important Rank 1 modification. 
$$H_{k+1}^{SR1} = H_k + \frac{(s_k - H_ky_k)(s_k - H_ky_k)^T}{(s_k - H_ky_k)^Ty_k}$$

\subsection{Cost Estimates}
In this subsection we estimate the cost by counting the call time of $f$ $g$ and $h$.
\paragraph{Line Search}
For each line search iteration, the cost is 
\begin{itemize}
	\item .618 method: \#f: 1, \#g: 0
	\item Armijo \& Goldstein: \#f:1, \#g:0
	\item (s-)Wolfe: \#f:1, \#g:1
\end{itemize}
The backtracking and interpolation will not create new call. 

\paragraph{Newton}
Each iteration costs: \#f:0, \#g:1, \#h:1

LM and mixed modification methods do not create new call. 

Quasi-newton method only costs \#h:1, and use \#g:1 each iteration.

However, in practice in our report, we use numerical gradient and numerical Hessian, using each $2n$ and $4n^2$ calls of f. Thus we only record the call time of f.
\section{Numerical Experiments}
In the rest of our report, we test several concrete problems. We use MATLAB to implement, running on my laptop with core I7-8700HQ.

\subsection{Problem Setting}
We test three type of problems. Let $n$ denote the dimension of the problem, and the first two problems has the type :$f = \sum_{i=1}^n f_i^2$.
\paragraph{Brown and Dennis Function}
$n= 4$, $f_i = (x_1 + t*x_2 - e^t)^2 + (x_3 + \sin(t)*x_4 + \cos(t))^2$, where $t = \frac{i}{5}$ for $i =1,\cdots, m$.

We choose the initial value $[25,5,-5,1]^T$ , and $m = 4,10,20,30,40,50$ throughout this paper.

\paragraph{Discrete Integral Equation}
$n = m$, $$f_i = x_i + \frac{h}{2}[(1-t_i)\sum_{j\le i}t_j(x_j + t_j + 1)^3 + t_i\sum_{j>i}(1-t_j)(x_j + t_j  +1)^3]$$
where $h = \frac{1}{n+1}, t_j = th$.

The initial value is chosen as $x_j = t_j(t_j-1)$.

\paragraph{Minimal Surface Equation}
We turn to seek a solution of the following minimal surface equation, 
$$\min_u{\int_{\Omega}\sqrt{1 + |\nabla u|^2} du}$$
with boundary condition $u(x,y)|D = x^2+y^2$. Here $\Omega$ is $[0,1]^2$ and $D$ is its boundary.
For a given square grid $U_{i,j} (0\le i,j \le n)$, we turn to minimize the following discrete functional.
$$\sum_{i=1}^n \sum_{j=1}^n (1 + (\frac{U_{ij}- U_{i-1,j}}{h})^2 + (\frac{U_{ij}- U_{i,j-1}}{h})^2)^{1/2} + \sum_{i=1}^n \sum_{j=1}^n (1 + (\frac{U_{ij}- U_{i+1,j}}{h})^2 + (\frac{U_{ij}- U_{i,j+1}}{h})^2)^{1/2}.$$
The initial value is randomly chosen in $N(0,1)$.

\subsection{Implementation Details}
In this subsection we show some implementation details, some of them might be discussed in the previous section and some of them are just to clarify our choice.

\begin{enumerate}
	\item Line Search Rule: We choose strong Wolfe, with $\rho = .3,~~\sigma = .6$. The max iteration is set as 100. Once the line search method fails, we turn to use exact line search (.618 method) to seek for step size.
	\item Modification: for damped Newton method, we use both "mixed" and "LM". For quasi-Newton method, we use only "mixed".
	
\end{enumerate}


\subsection{Numerical Result}
In this subsection we test our algorithms in the problems. We first show the power of damped Newton method. We use ``bd[m]'' to denote the Brown and Dennis function with size $m$, and likewise we use ``die[m]'' and ``mse[m]''. 

We record the iteration, subiteration, the call of $f$ (including numerical gradient and hessian) and the value to test whether our algorithm is successful.
\begin{table}[H]
\footnotesize
	\begin{tabular}{|c|c|c|c|c|c|c|c|c|c|c|c|c|c|c|c|}
		\hline
		& \multicolumn{5}{c|}{NONE}                & \multicolumn{5}{c|}{MIXED}               & \multicolumn{5}{c|}{LM}                   \\ \hline
		& iter & sub  & time(s) & \#f   & value    & iter & sub  & time(s) & \#f   & value    & iter & sub  & time(s) & \#f    & value    \\ \hline
		bd4  & 18   & 533  & 0.19    & 10113 & 1.05E-05 & 19   & 607  & 0.19    & 13151 & 1.05E-05 & 22   & 707  & 0.21    & 15081  & 1.05E-05 \\ \hline
		bd10 & 10   & 319  & 0.09    & 5188  & 1.44E+00 & 11   & 396  & 0.13    & 7117  & 1.44E+00 & 11   & 432  & 0.14    & 7347   & 1.44E+00 \\ \hline
		bd20 & 12   & 415  & 0.10    & 5118  & 8.58E+04 & 9    & 312  & 0.09    & 5556  & 8.58E+04 & 47   & 1632 & 0.52    & 28900  & 8.58E+04 \\ \hline
		bd30 & 16   & 710  & 0.21    & 8377  & 9.77E+08 & 28   & 1061 & 0.41    & 20845 & 9.77E+08 & 18   & 620  & 0.18    & 9676   & 9.77E+08 \\ \hline
		bd40 & 16   & 744  & 0.15    & 7082  & 5.86E+12 & 71   & 2793 & 0.98    & 54692 & 5.86E+12 & 59   & 2265 & 0.78    & 42129  & 5.86E+12 \\ \hline
		bd50 & 46   & 1798 & 0.61    & 32616 & 2.67E+16 & 16   & 721  & 0.21    & 11832 & 2.67E+16 & 239  & 9247 & 3.40    & 171884 & 2.67E+16 \\ \hline
	\end{tabular}
\\
\\
	\begin{tabular}{|c|c|c|c|c|c|c|c|c|c|c|c|c|c|c|c|}
		\hline
		& \multicolumn{5}{c|}{NONE}                 & \multicolumn{5}{c|}{MIXED}               & \multicolumn{5}{c|}{LM}                  \\ \hline
		& iter & sub & time(s)  & \#f    & value    & iter & sub & time(s) & \#f    & value    & iter & sub & time(s) & \#f    & value    \\ \hline
		die2  & 11   & 245 & 0.14& 4565   & 8.88E-20 & 11   & 245 & 0.14    & 4565   & 8.88E-20 & 11   & 245 & 0.13    & 4565   & 8.88E-20 \\ \hline
		die10 & 12   & 299 & 0.44     & 16848  & 2.83E-18 & 12   & 299 & 0.46    & 16848  & 2.83E-18 & 12   & 299 & 0.49    & 16848  & 2.83E-18 \\ \hline
		die20 & 13   & 383 & 1.39     & 42879  & 1.87E-18 & 12   & 295 & 1.29    & 40361  & 2.58E-18 & 12   & 350 & 1.43    & 40771  & 1.29E-18 \\ \hline
		die30 & 12   & 349 & 2.50     & 62727  & 2.71E-18 & 11   & 293 & 2.30    & 59142  & 1.82E-18 & 12   & 312 & 2.94    & 74996  & 4.53E-19 \\ \hline
		die40 & 15   & 390 & 6.79     & 144560 & 9.30E-19 & 13   & 336 & 5.47    & 116936 & 7.56E-20 & 13   & 387 & 5.49    & 119199 & 6.14E-18 \\ \hline
		die50 & 17   & 479 & 12.56    & 225927 & 3.84E-18 & 10   & 254 & 7.69    & 137170 & 4.22E-18 & 13   & 355 & 9.92    & 178765 & 3.46E-18 \\ \hline
	\end{tabular}

	\begin{tabular}{|c|c|c|c|c|c|c|c|c|c|c|c|c|c|c|c|}
		\hline
		& \multicolumn{5}{c|}{NONE}                 & \multicolumn{5}{c|}{MIXED}               & \multicolumn{5}{c|}{LM}                  \\ \hline
		& iter & sub & time(s)  & \#f    & value    & iter & sub & time(s) & \#f    & value    & iter & sub & time(s) & \#f    & value    \\ \hline
		mse3  & 19   & 112 & 0.14 & 2971   & 1.46E+00 & 19   & 112 & 0.14    & 2971   & 1.46E+00 & 19   & 112 & 0.12    & 2971   & 1.46E+00 \\ \hline
		mse5  & 17   & 68  & 0.73     & 21029  & 2.08E+00 & 14   & 35  & 0.62    & 17079  & 2.08E+00 & 16   & 70  & 0.90    & 22715  & 2.08E+00 \\ \hline
		mse7  & 18   & 175 & 3.94     & 111099 & 2.40E+00 & 18   & 71  & 3.58    & 101257 & 2.40E+00 & 16   & 116 & 3.38    & 97620  & 2.40E+00 \\ \hline
	\end{tabular}
\caption{Damped Newton Method with several modifications.}
\end{table}

\begin{table}[H]
	\centering
	\begin{tabular}{|c|c|c|c|c|c|c|c|c|c|c|}
		\hline
		& \multicolumn{5}{c|}{$\varphi = 0$}             & \multicolumn{5}{c|}{$\varphi = 1/2$}           \\ \hline
		& iter & sub    & time(s)  & \#f     & value     & iter & sub    & time(s)  & \#f     & value     \\ \hline
		bd4  & 1484 & 51901  & 20.55242 & 1383707 & 1.05E-05  & 1381 & 47958  & 20.44    & 1466973 & 1.05E-05  \\ \hline
		bd10 & 2000 & 80026  & 27.54    & 1678135 & 1.44E+00  & 5530 & 221197 & 63.52    & 3639777 & 1.44E+00  \\ \hline
		bd20 & 1617 & 64644  & 18.35    & 919048  & 8.58E+04  & 66   & 2645   & 0.80     & 43293   & 8.58E+04  \\ \hline
		bd30 & 170  & 6832   & 2.09     & 107212  & 9.77E+08  & 70   & 2831   & 0.88     & 45195   & 9.77E+08  \\ \hline
		bd40 & 234  & 9407   & 2.92     & 148971  & 5.86E+12  & 106  & 4291   & 1.53     & 72682   & 5.86E+12  \\ \hline
		bd50 & 182  & 7384   & 2.97     & 119346  & 2.67E+16  & 836  & 33341  & 13.63    & 588864  & 2.67E+16  \\ \hline
		& \multicolumn{5}{c|}{$\varphi = 1$}             & \multicolumn{5}{c|}{SR1}                       \\ \hline
		bd4  & 2814 & 110293 & 48.46304 & 3477151 & 1.05E-05  & 899  & 31943  & 9.540389 & 555412  & 1.05E-05  \\ \hline
		bd10 & 332  & 13311  & 3.732782 & 213557  & 1.4432255 & 1222 & 48885  & 13.79036 & 768891  & 1.4432255 \\ \hline
		bd20 & 47   & 1885   & 0.507918 & 25765   & 85822.202 & 1136 & 45439  & 14.88648 & 712013  & 85822.202 \\ \hline
		bd30 & 110  & 4431   & 1.596179 & 73979   & 976882218 & 128  & 5149   & 1.490632 & 74972   & 976882218 \\ \hline
		bd40 & 60   & 2397   & 1.121881 & 38197   & 5.856E+12 & 212  & 8507   & 2.992672 & 150455  & 5.856E+12 \\ \hline
		bd50 & 523  & 20950  & 10.44849 & 380114  & 2.67E+16  & 510  & 20436  & 7.583633 & 333912  & 2.67E+16  \\ \hline
	\end{tabular}
\caption{Broyden's family for Brown and Dennis function, choosing DFP, BFGS, $\varphi = 1/2$ and SR1.}
\end{table}
\begin{table}[H]
	\centering
	\begin{tabular}{|c|c|c|c|c|c|c|c|c|c|c|}
		\hline
		& \multicolumn{5}{c|}{$\varphi = 0$}         & \multicolumn{5}{c|}{$\varphi = 1/2$}       \\ \hline
		& iter & sub  & time(s)  & \#f    & value    & iter & sub  & time(s)  & \#f    & value    \\ \hline
		die2  & 15   & 368  & 0.288981 & 5997   & 1.17E-18 & 12   & 248  & 0.14     & 3927   & 2.64E-18 \\ \hline
		die10 & 48   & 1815 & 2.10     & 71406  & 6.36E-17 & 55   & 2095 & 2.45     & 82235  & 1.49E-17 \\ \hline
		die20 & 53   & 1941 & 4.21     & 123978 & 1.15E-16 & 61   & 2298 & 5.19     & 152758 & 5.76E-17 \\ \hline
		die30 & 48   & 1812 & 8.43     & 212699 & 9.31E-17 & 58   & 2173 & 7.30     & 183230 & 2.55E-16 \\ \hline
		die40 & 49   & 1853 & 12.99    & 249706 & 7.68E-17 & 51   & 1933 & 13.11    & 260561 & 1.32E-16 \\ \hline
		die50 & 21   & 733  & 5.58     & 97410  & 1.45E-16 & 48   & 1813 & 16.96    & 295933 & 2.40E-16 \\ \hline
		& \multicolumn{5}{c|}{$\varphi = 1$}         & \multicolumn{5}{c|}{SR1}                   \\ \hline
		die2  & 22   & 686  & 0.357146 & 10552  & 4.30E-18 & 15   & 368  & 0.18309  & 6002   & 4.28E-18 \\ \hline
		die10 & 64   & 2455 & 3.034302 & 99611  & 7.06E-17 & 32   & 1175 & 1.420462 & 46141  & 1.78E-17 \\ \hline
		die20 & 55   & 2058 & 4.215484 & 125558 & 1.88E-16 & 36   & 1261 & 2.794558 & 79110  & 4.31E-18 \\ \hline
		die30 & 62   & 2372 & 9.121304 & 233425 & 2.20E-16 & 31   & 1132 & 4.239021 & 112188 & 1.31E-17 \\ \hline
		die40 & 67   & 2573 & 14.63883 & 306334 & 2.31E-16 & 33   & 1213 & 8.042353 & 170885 & 2.21E-17 \\ \hline
		die50 & 17   & 573  & 7.425253 & 130492 & 3.16E-18 & 25   & 893  & 9.095294 & 149673 & 8.53E-17 \\ \hline
	\end{tabular}
\caption{Broyden's family for Discrete Integration Equation, choosing DFP, BFGS, $\varphi = 1/2$ and SR1.}
\end{table}
\begin{table}[H]
	\centering
	\begin{tabular}{|c|c|c|c|c|c|c|c|c|c|c|}
		\hline
		& \multicolumn{5}{c|}{$\varphi = 0$}          & \multicolumn{5}{c|}{$\varphi = 1/2$}       \\ \hline
		& iter & sub   & time(s) & \#f     & value    & iter & sub   & time(s) & \#f    & value    \\ \hline
		mse3 & 871  & 1831  & 1.75    & 36299   & 1.46E+00 & 772  & 1606  & 1.63    & 32186  & 1.46E+00 \\ \hline
		mse5 & 5151 & 15436 & 29.50   & 718067  & 2.08E+00 & 5367 & 16089 & 31.51   & 748101 & 2.08E+00 \\ \hline
		mse7 & 3624 & 10831 & 46.44   & 1094006 & 2.40E+00 & 1646 & 4907  & 21.16   & 502604 & 2.40E+00 \\ \hline
		& \multicolumn{5}{c|}{$\varphi = 1$}          & \multicolumn{5}{c|}{SR1}                   \\ \hline
		mse3 & 837  & 1731  & 1.68    & 34841   & 1.46E+00 & 765  & 1541  & 1.51    & 31531  & 1.46E+00 \\ \hline
		mse5 & 2568 & 7680  & 14.35   & 359115  & 2.08E+00 & 1123 & 3330  & 6.83    & 158131 & 2.08E+00 \\ \hline
		mse7 & 1212 & 3624  & 17.15   & 373014  & 2.40E+00 & 800  & 2347  & 9.46    & 249606 & 2.40E+00 \\ \hline
	\end{tabular}
\caption{Broyden's family for Minimal Surface Equations, choosing DFP, BFGS, $\varphi = 1/2$ and SR1.}
\end{table}

\section{Conclusion}
In this lab we test the Newton method and Quasinewton method in several types of problems. As we see, under appropriate modification, the newton method is powerful with fewer iteration and subiteration. For the brown and dennis function, we found that the quasi newton is less-efficient, since it might be degenerate to the gradient type method. However, we find in the DIE problem the quasi newton is powerful, especially DIE50. The observation shows that the behaviour of Newton type method is heavily dependent on our modification technique. Otherwise it is likely to fail. Also, for the MSE problem, we find that the method will get stuck in some local minima, Newton type method cannot avoid this case.
\end{document}
















Escape special TeX symbols (%, &, _, #, $)